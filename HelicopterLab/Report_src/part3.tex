\section{Multivariable Control}

Using the system described by (6a)-(6b) we get the following state-space model
\begin{equation}
    \vec{\dot{x}} = \begin{bmatrix}     % A
                                0&1&0\\
                                0&0&0\\
                                0&0&0
                    \end{bmatrix}
                    \begin{bmatrix}     % x
                                \tilde{p}\\
                                \dot{\tilde{p}}\\
                                \tilde{e}
                    \end{bmatrix}
                    +
                    \begin{bmatrix}     % B
                                0&0\\
                                0&K_1\\
                                K_2&0
                    \end{bmatrix}
                    \begin{bmatrix}     % u
                                \tilde{V_s}\\
                                \tilde{V_d}
                    \end{bmatrix}
\end{equation}
%
\subsection{Controllability and LQR}
The controllability matrix for a $3\times3$ system is given by
\begin{equation}
    \begin{bmatrix}
            B & AB & A^2B
    \end{bmatrix}
    =
    \begin{bmatrix}
            0&0&0&K_1&0&0\\
            0&K_1&0&0&0&0\\
            K_2&0&0&0&0&0
    \end{bmatrix}
\end{equation}
which has $ rank = n = 3 $. Thus the system is controllable.\\ \\
Now we will use a controller on the form $$ \vec{u} = \vec{Pr}-\vec{Kx}$$
where the input $\vec{u} = -\vec{Kx}$ minimizes the costfunction 
$$J = \int_{0}^{\infty}(\vec{x}^T(t)\vec{Qx}(t) + \vec{u}^T(t)\vec{Ru}(t))dt $$ and $\vec{r} = [\tilde{p}_c \quad \dot{\tilde{e}}_c]^T$. $\vec{K}$ here corresponds to the linear quadratic regulator and $\vec{Q}$ and $\vec{R}$ are diagonal matrices. \\

The matrix $\vec{P}$ is found such that $\lim_{t\rightarrow\infty}\tilde{p}=\tilde{p}_c$ and $\lim_{t\rightarrow\infty}\dot{\tilde{e}} = \dot{\tilde{e}}_c$:
\begin{align}
    \dot{\vec{x}} &= \vec{Ax} + \vec{Bu}    \nonumber\\ \nonumber
    &= \vec{Ax} + \vec{B(Pr-Kx)} = 0\\\nonumber
    \Rightarrow& (\vec{A}-\vec{BK})\vec{x}_\infty = -\vec{BP}\vec{r}_0\\\nonumber
    \Rightarrow& \vec{y}_\infty = [\vec{C}(\vec{BK}-\vec{A})^{-1}\vec{B}]\vec{Pr}_0 \nonumber\\ \nonumber\\
    \text{which in turn gives}\nonumber\\ \nonumber\\
    \vec{P} &= [\vec{C}(\vec{BK}-\vec{A})^{-1}\vec{B}]^{-1} \label{Pmatrix}
\end{align}



Our weighting matrices were as follows
%
\begin{align*}
    \vec{Q} = \begin{bmatrix}
                15&0&0\\
                0&1&0\\
                0&0&10
            \end{bmatrix} 
    \quad
    \vec{R} = \begin{bmatrix}
                0.1&0\\
                0&0.3
            \end{bmatrix}
\end{align*}
%
The elements of the matrix $\vec{Q}$ were chosen such that it penalized deviations in the pitch and elevation rate, i.e. $\vec{Q}_{11}$ and $\vec{Q}_{33}$ were increased. $\vec{Q}_{11} < \vec{Q}_{33}$ gave less elevation control from the helicopter, thus we chose $\vec{Q}_{11} > \vec{Q}_{33}$ as shown. \\
The elements of R were set a bit lower as to "lower the cost" of using input. $\vec{R}_{22}$ were chosen larger than $\vec{R}_{11}$ as the opposite gave o.k. control, but resulted in a trade-off in regard to elevation vs pitch control. \\

The resulting $\vec{K}$ matrix were found using the MATLAB-command 
%
\begin{lstlisting}  
    K = lqr(A, B, Q, R);
\end{lstlisting}
%
then the $\vec{P}$ matrix were calculated using \cref{Pmatrix}, giving
%
\begin{align*}
    \vec{K} =   \begin{bmatrix}
                    0       & 0      & 10 \\
                    7.07    & 5.12   & 0
                 \end{bmatrix}
        \quad \quad
    \vec{P} =   \begin{bmatrix}
                    0       & 10 \\
                    7.07    & 0
                \end{bmatrix} .
\end{align*}
%
%
%
%\newpage
\subsection{LQR with PI}

To include a PI-controller in our model we introduce two new states 
$$  \dot{\gamma} = \tilde{p} - \tilde{p}_c $$ 
$$  \dot{\zeta} = \tilde{e} - \tilde{e}_c  $$
which gives a new state vector 
$\vec{x} = [\tilde{p} \enskip \dot{\tilde{p}} \enskip \dot{\tilde{e}} \enskip \gamma \enskip \zeta]^T$. This in turn gives the new state-space matrices
\begin{align*}
    \vec{A} =   \begin{bmatrix}
                    0&1&0&0&0&0\\
                    0&0&0&0&0&0\\
                    0&0&0&0&0&0\\
                    1&0&0&0&0&0\\
                    0&0&1&0&0&0
                \end{bmatrix}
        \quad
    \vec{B} =   \begin{bmatrix}
                    0&0\\
                    0&K_1\\
                    K_2&0\\
                    0&0\\
                    0&0
                \end{bmatrix}
\end{align*}
as well as a new $\vec{Q}$ matrix 
$$\vec{Q} = \begin{bmatrix}
                15&0&0&0&0\\
                0&1&0&0&0\\
                0&0&10&0&0\\
                0&0&0&2&0\\
                0&0&0&0&10
            \end{bmatrix}.
$$
This state expansion also increases the dimensions of $\vec{K}$, which is calculated as before. $\vec{P}$ can't now be calculated as before, but observing that 
$$
\vec{K} =   \begin{bmatrix}
                \vec{K}{11} & ... & \vec{K}_{1m}\\
                .&.&.\\
                .&.&.\\
                .&.&.\\
                \vec{K}_{n1} & ... & \vec{K}_{nm}
            \end{bmatrix}
    \quad \Rightarrow \quad
\vec{P} =   \begin{bmatrix}
                \vec{K}_{11} & \vec{K}_{1m}\\
                \vec{K}_{n1} & \vec{K}_{nm}
            \end{bmatrix}
$$
throughout, simplifies the calculations for $\vec{P}$.\\
The PI controller performed better overall than the P controller. With the PI we were able to get faster to steady-state without overshooting as well as getting proper elevation control with constant input.